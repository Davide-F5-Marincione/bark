\documentclass[twocolumn, a4paper]{scrartcl}

\usepackage{microtype}
\usepackage[scale=.8]{geometry}
\usepackage{graphicx}


\title{Bark: An Attributed Graph Grammar Chess Engine}
\author{Davide Marincione}

\begin{document}
    \maketitle
    \section{Introduction}
    Chess is a fairly complex game, with a relatively wide set of rules and a small but non-trivial branching factor (unlike Go, which has very simple rules but huge branching factor). This makes it a good candidate for projects such as Bark. Unlike traditional chess engines, Bark is not programmed in any language: it is instead a set of rules that can be applied to a graph, which is then used to play chess. This makes it a very complex and unwieldy project, as it requires a lot of work to define the rules and the graph structure. In this report, I will describe the rules and the graph structure used in Bark, and I will discuss the challenges and limitations of the project.

\end{document}

